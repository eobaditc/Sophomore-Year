\documentclass{article} 

\usepackage[utf8]{inputenc} 

\title{Homework 06: Document Tools} 

\author{Emily Obaditch} 
\date{March 18 2016} 
\usepackage{natbib} 
\usepackage{graphicx} 
\usepackage[margin=1in]{geometry} 

\begin{document} 

\maketitle 

\section*{Overview}
    For this assignment I chose to represent the distribution of gender 
/ ethnicity using percentages rather than raw numbers.  I chose to do 
this because there are different amounts of students per year and the 
histogram could looked skewed if not all in the same scope.  I first 
processed the \verb|demographics.csv| file with the 
\verb|demographics.sh| shell script.  This script sums the total of each 
catagory for each year (i.e. Male, Female, Caucasion, Oriental, etc...)  
From this I used two other shell scripts \verb|race.sh| and 
\verb|gender.sh| to calculate percentages of each category.  With this 
data I then used gnuplot to graph clustered histograms with a legend.  
Each shell script \verb|race.sh| and \verb|gender.sh| create a 
\verb|resultsRace.dat| or \verb|resultsGender.dat| file which is how I 
obtained values for the table.
    In doing this assignment I noticed that while the gender 
distribution leans strongly towards male, the ethnic distribution is in 
similar accordance with that of the university.  As the years have gone 
by there is a slight increase in the female population of Computer 
Science majors.  I learned that data can be represented in many 
different ways.  I found that the easiest way to see trends among groups 
of people was a histogram as it graphics multiple categories on one 
figure.  My main takeaway from this assignment is that the major is 
lacking in diversity however showing a positive trend for future years.
    
\section*{Methodology}
    I first processed the \verb|demographics.csv| file using the script 
\verb|demographics.sh|.  This script uses a for loop to go through each 
field in the file (using a comma as the delimiter).  The for loop 
changes that flag used in the cut command.  The script counts the amount 
of a certain character using grep and wc -l.  A variable is set for each 
gender and ethnicity and the totals are displayed at the end of each 
iteration.  For example, when counting how many student were male the 
line:
    \begin{verbatim}
    Male=$(cat demographics.csv | cut -d , $cFlag | grep -o 'M' | wc-l)
    \end{verbatim}
    The final output of \verb|demographics.sh| is the total number of 
each category for each year, separated by commas.  Next, I wrote two 
more scripts \verb|gender.sh| and \verb|race.sh|.  Both of these scripts 
call \verb|demographics.sh|. The \verb|gender.sh| script deletes the 
first row of output from \verb|demographics.sh| which contains the 
category titles.  I then used cut and awk to calculate the percentages 
of male and female for every year using the following line:
    \begin{verbatim}
        ./demographics.sh | sed 1d | cut -d , -f 1-3 | awk 'BEGIN { FS = 
"," };
                {percentM=100* ( $2 / ($3 + $2))};
                {percentF=100* ( $3 / ($3 + $2))};
                {print $1"\t"percentM"\t"percentf}' > resultsGender.dat
    \end{verbatim}
        The output is sent to resultsGender.dat to be used later when 
plotting.
    
    
    \verb|Race.sh| uses the same method but with more categories.  The 
command used in this script is:
    \begin{verbatim}
        ./demographics.sh | sed 1d| cut -d , -f 1,4-10 | awk 'BEGIN { FS 
= "," };
            { sum = $2+$3+$4+$5+$6+$7+$8};
            { percentC=100* ( $2 / sum)};
            { percentO=100* ( $3 / sum)};
            { percentS=100* ($4 / sum)};
            { percentB=100* ($5 / sum)};
            { percentN=100* ($6/sum)};
            { percentM=100* ($7 / sum)};
            { percentU=100* ($8 / sum)};
            { print $1 "\t" percentC "\t" percentO "\t" percentS "\t" 
percentB
    "\t" percentN "\t" percentM "\t" percentU }' > resultsRace.dat
    \end{verbatim}
        The output is sent to resultsRace.dat to be used later when 
plotting.
    
    Gnuplot is used to create histograms for the gender distribution and 
the ethnicity distribution.  The style is set to clustered histogram 
which creates a histogram of multiple bars.  Each year is used as the 
xticlabel.  A key is set using the command \verb|set key top right|.  
The legend labels are create by specifying the title for each line after 
the plot command is run, i.e:
    \begin{verbatim}
        plot 'resultsGender.dat' using 2:xticlabels(1) title "Male", '' 
using 3:xticlabels(1) title "Female"
    \end{verbatim}
        A similar command is used for creating the ethnicity histogram.
        
\section*{Analysis}
    From looking at Tables 1-2 and Figures 1-2 it can be concluded that 
the majority of the Computer Scientists at Notre Dame are male and 
Caucasian.  As seen in Table 1, the range of male percentage is 71-78\% 
and the range of female percentage is 22-28\%.  The histogram in Figure 
1 shows that as the years progress the distribution becomes more and 
more even, a trend that I hope to see continue as a woman in the field.
    
    The ethnic diversity in the Computer Science Engineering major is 
surprising.  As seen in the histogram the majority of the major for 
every year is Caucasian.  The second most prominent ethnicity's are 
Oriental and Hispanic.  Table 2 shows the exact percentages.  In most 
years 0\% of Computer Science majors identified as undeclared or 
multiple.  Figure 2 and Table 2 bring to light the lack of diversity 
within the major however show an improving trend. \begin{table}[h!]
    \centering
    \begin{tabular}{|c||c|c|}\hline
     Year & Male & Female\\\hline
     2013 & 77.8 & 22.2 \\\hline
     2014 & 78.6 & 21.4\\\hline
     2015 & 78.4 & 21.6\\\hline
     2016 & 75.9 & 24.1\\\hline
     2017 & 71.4 & 28.6\\\hline
     2018 & 71.4 & 28.6\\\hline
    \end{tabular}
    \caption{Gender Distribution (\%)}
    \label{tab:my_label} \end{table} \begin{table}[h!]
    \centering
    \begin{tabular}{|c||c|c|c|c|c|c|c|}\hline
        Year & Caucasian & Oriental & Hispanic & African American & 
Native American & Multiple & Undeclared\\\hline
        2013 & 68.2&11.1&11.1&4.7&1.6&3.2&0\\\hline
        2014 & 76.8&8.9&7.1&3.5&1.78&1.78&0\\\hline
        2015 & 63.5&12.2&13.5&5.4&1.4&1.4&2.7\\\hline
        2016 & 67 & 11.4 & 11.4 & 1.3 & 8.8 & 0&0\\\hline
        2017 & 65.9& 13.2&3.3&5.5&5.5&6.6&0\\\hline
        2018 & 72.2& 6.4& 9.5 & 2.4 & 32 & 6.4& 0 \\\hline
    \end{tabular}
    \caption{Ethnic Distribution (\%)}
    \label{tab:my_label} 
    \end{table} 
    \begin{figure}[H] 
        \centering 

        \includegraphics[width=5in]{resultsGender.png} \caption{Gender 
Distribution} 
        \label{fig:gender} 
        \end{figure} 

\begin{figure}[H] 
    \centering 
    \includegraphics[width=5in]{resultsRace.png} 
    \caption{Ethnicity Distribution} 
    \label{fig:ethnicity} 
\end{figure} 

\section*{Discussion}
    As a woman, the issues of gender and ethnicity are very important to 
me.  I myself have never been intimidated by the strong male presence in 
the engineering field however still believe that more girls should be in 
the program.  I think that the department should try to increase 
diversity (both gender and ethnicity) through out reach programs to 
young girls and inner cities.  I think exposure to Computer Science and 
Engineering at a young age could do wonders for the diversity in the 
field.  As for the industry, I do not think that a firm should hire a 
computer scientist solely on ethnicity or gender.  I think that if 
universities get better diversity in the major then there will only be 
competition for jobs based on skill rather than on race or gender.
    I have experienced a very welcoming environment fro the department 
of Computer Science and Engineering.  Each faculty member is wholly 
inclusive no mater what ethnicity or gender a student it.  However I 
have noticed that when choosing groups male students tend to gravitate 
towards other males and will only be in a group with girls if they know 
the girl and know that she "is smart enough."  I find this extremely 
problematic.  This is really the only challenge I have faced so far in 
terms of diversity.  This can be improved by making sure groups are well 
balanced or assigning project groups instead of letting the students 
pick.  Overall though I do believe that the department provides a 
welcoming environment for all types of students. 
\end{document}
